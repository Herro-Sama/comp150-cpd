% Please do not change the document class
\documentclass{scrartcl}

% Please do not change these packages
\usepackage[hidelinks]{hyperref}
\usepackage[none]{hyphenat}
\usepackage{setspace}
\doublespace

% You may add additional packages here
\usepackage{amsmath}

% Please include a clear, concise, and descriptive title
\title{CPD Report}

% Please do not change the subtitle
\subtitle{COMP160 - CPD Tasks}

% Please put your student number in the author field
\author{1507729}

\begin{document}

\maketitle

\section{Introduction}

During my second semester I have found multiple areas which I can improve. I would like to make sure that I have much less unnecessary downtime along with spending more time practising my core coding skills, expanding my knowledge of networking and servers and finally look into ways to keep my self motivated for extended periods of time.  I will further expand upon these points showing how I plan to improve upon them and the time frame for which I expect to be able to have improved. I will then have a short conclusion outlining how I expect the improvements will assist me going forward and the repercussions of these changes.

\section{Reduce Unnecessary Downtime}

I need to stay more involved during group projects, especially during the formative stages. It should be quite clear if I have been able to reduce my downtime, as I will have considerably more work created for my next project. I think that this is possible to reduce my unnecessary downtime, it just requires me to focus and do the work. If I am to be able to work independently of others, or within any kind of team I will need to be able to focus and get work completed. This is something I should be able to resolve immediately however it will be very clear and apparent going forward if I am able to cut down on unnecessary wasted time. 


\section{Networking \& Servers}

I think I need to spend more time learning about servers and networking in preparation for next semesters module. If I have reached an acceptable stage with this I should be able to create and setup a server from scratch without having outside or at least minimal outside assistance. I think that it's perfectly reasonable to assume that I should be able to create a server independently as all the resources are available for me to use. This should have a direct impact on my 2nd semester as I should be able to work autonomously in the creation of the server, it will also be a useful skill to have going forward in my coding career. I will obviously need to work on improving these skills before the beginning of my next semester so I am ready for when this particular skill is required. 


\section{Additional Practice Coding in Unreal}

I am still relatively new to coding C++ in Unreal and would like to further develop my skills and understanding of the different functions and names within Unreal. I will be able to create C++ classes and perform various functions within those classes I would like to spend at least five hours a week on this. I think that it is very possible and I should be able to to learn the various systems within Unreal through the numerous online resources. As coding is an incredibly sought after skill, and the one I am currently developing for my career I think it speaks for it's self as the the future applications. I would like to spend time during the run-up to the 2nd semester getting more acquainted with coding for Unreal Engine. 



\section{Clean Commented Code}

During my time in working with blueprints there have been several occasions when I feel that the readability of my code/blueprints would have been aided had I opted to simply comment my code to make it more readable and understandable. This goal should be immediately apparent when looking at my work as it will be clearly coded and cleaned up to remove unnecessary parts, clutter and will be commented for ease of reading. This is the level of which I will be expected to work and is something I should be doing instinctively rather than through prompting. This falls under key skills I should be expanding to make myself ready for a career in coding. I would like to expand on my abilities at the same time I am expanding my knowledge of coding as mentioned above.


\section{Motivation}

I feel like my motivation for my group project has dipped significantly, this is a problem that I need to address for future projects to make sure that I am working in peak condition. If my motivation is at a continued high, I will hopefully be showing that in my sprint retrospectives and in my work as it should be of a higher quality. Although humans don't always have full control over their own motivation, there are strategies which can help to reduce emotional fatigue and raise morale. I think this will be relevant as working on a projects that could potentially take years to complete, I will be required to stay motivated and have a maintainable standard of work. Although this is going to be an ongoing exercise I feel like this is something I should focus on ready for my next long group project or at least have a plan in place ready for it. 

\section{Conclusion}

This second semester has gone relatively well however, I do feel like there is a significant area for improvement and a lot of valuable insights were learned from this experience. I believe I need to also expand upon my team working abilities and I will hopefully be able to utilise more of my particular skill set going forward to really help aid the design and creation of the next project. It is my understanding that I have some key areas to improve within but I don't think that these are areas of extreme concern.

\end{document}
