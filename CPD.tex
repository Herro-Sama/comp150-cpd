% Please do not change the document class
\documentclass{scrartcl}

% Please do not change these packages
\usepackage[hidelinks]{hyperref}
\usepackage[none]{hyphenat}
\usepackage{setspace}
\doublespace

% You may add additional packages here
\usepackage{amsmath}

% Please include a clear, concise, and descriptive title
\title{CPD Report}

% Please do not change the subtitle
\subtitle{COMP150 - CPD Tasks}

% Please put your student number in the author field
\author{1507729}

\begin{document}

\maketitle

\section{Introduction}

During my first semester I have identified multiple areas for improvement with regards to my core understanding of coding constructs as well as terminology, communication within a team including utilising Agile effectively and independent industry related research. I will further expand upon these points showing how I plan to improve upon them and the time frame for which I expect to be able to have improved. I will then have a short conclusion outlining how I expect the improvements will assist me going forward and the repercussions of these changes.

\section{Area's of Improvement}

I need to reinforce my understanding of coding terminology. This has become evident as my understanding of the terminology is sub-standard, and in working to increase this understanding I will be able to hold more productive conversations with peers. A decent understanding of coding terminology such as what a compound statement is and other terms of similar nature, is expected of a professional programmer. This is an area that I will have to work on but is easily rectified simply by asking questions or enquiring into this terms. It's important that I have this skill as people within industry are expected to know this information and it's a key area that I need to fully establish if I wish to excel within my desired field. I would like to have this knowledge before the end of my first year, to prevent conflicts when discussing with peers and improve my understanding of their suggested solutions. I expect that I will not be able to gather all nuances but a firmer grasp will be important and a greater core understanding should be a more than reasonable goal for the end of the year. \linebreak

I will need to work on improving my ability to create code rather than just adapting existing code to suit my needs. This should be directly reflective in my work, showing a greater level of problem solving and understanding along with fewer references to others existing code. I don't expect to be able to work totally autonomously but I hope to be able to build upon my knowledge of coding constructs and solutions and to then be able to approach challenges in a more independent fashion. This is a very broad goal but to a more refined extent, even being able to work independently on smaller tasks initially is a key quality I would like to build upon going forward, and I feel is expected of anyone working as a programmer within the games industry. I would like to be able to do more independent coding by the end of the first year however this is a skill that I will be constantly working to develop and improve well beyond University. \linebreak

I need to work on my ability to communicate between peers especially with regards to version control as this has created multiple file conflict within my team projects. I would like to see a lower number of conflicts when merging my work with peers, to ensure that less time is wasted when implementing features or making changes to existing features. I think that this is achievable on a personal level ensuring that I keep my fellow teammates informed of changes I am making and being a more effective communicator within the team. This is something that I will be working on and is a highly important skill to have within industry I understand that file conflicts can happen quite easily and that it's almost impossible to remove them but by improving my communication abilities I hope to at least reduce the number on file conflicts created by work. I would ideally like to be able to look back at my next project and say that I have improved my communication skills and reduced the number of file conflicts, however I will be continuing to review my performance within this area to make sure that I am improving my communication skills constantly going forward from project to project. \linebreak

I would like to increase the amount of academic and industry related research that I carry out individually and improve my knowledge of the current industry practices. I think that it should be evident in my understanding of various topics within industry and familiarising myself with computing works should help me with problem solving and should be reflective within my work and with an immediate reflective look at the industry works that I have explored. This is something I should be able to do with the access I have to both the GDC vault and the IEEE archives. The only limiting factor will be the amount of time I am willing to invest into understanding the articles. I would like to be able to do more reading and build upon my limited understanding of current works especially within computing and computer research and with the access to resources I have very few limiting factors to prevent me from fully exploring this. I am going to aim to have at least four additional papers excluding asssigned reading and to listen to at least two talks within the GDC vault before the end of the first year and to then build on this number for each of the following semesters. \linebreak

I need to improve my understanding and ability to implement  the Agile philosophy within a team. It's my belief that working on improving my understanding of agile and making sure that I adhere to the methodologies being used by my team will help to improve my overall efficiency and should be reflective in the productivity I display within my group.This is something that will take considerable time to fully implement but working on key area's, like making sure that any daily stand ups I am apart of are succinct and relevant.This isn't unreasonable to suggest that I improve upon this skill set as I will need to make sure that I am a productive member of a team if I wish to be employable within the future. I would like to have improved in my ability to implement and be apart of an agile team by the time that I have finished first year however I will still need more experience working with a team to fully to adapt and implement Agile within the gaming industry.


\section{Conclusion}

I would like to see a gradual improvement to my problem solving, working efficiency, team communication and  overall employability. I will be continuing to build upon the five areas outlined above in a sustained effort to improve my overall ability to function not just as a productive member of a team but also as an individual worker. It's my belief that by implementing these changes and building upon them I will be able to not only improve my own productivity but also become a better example of industry practice assisting other team members to improve their performance as well.

\end{document}
